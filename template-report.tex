\documentclass[12pt, a4paper]{memoir} % for a short document
\usepackage[french,english]{babel}

\usepackage [vscale=0.76,includehead]{geometry}                % See geometry.pdf to learn the layout options. There are lots.
%\geometry{a4paper}                   % ... or a4paper or a5paper or ... 
%\geometry{landscape}                % Activate for for rotated page geometry
%\OnehalfSpacing
% \setSingleSpace{1.05}
%\usepackage[parfill]{parskip}    % Activate to begin paragraphs with an empty line rather than an indent


%===================================== packages
\usepackage{lipsum}
\usepackage{graphicx}
\usepackage{amsmath}
\usepackage{fullpage}
\usepackage{mathptmx} % font = times
\usepackage{helvet} % font sf = helvetica
\usepackage[latin1]{inputenc}
\usepackage{relsize}
\usepackage[T1]{fontenc}
\usepackage{tikz}
\usepackage{booktabs}
\usepackage{textcomp}%textquotesingle
\usepackage{multirow}
\usepackage{pgfplots}
\usepackage{url}
\usepackage{footnote}
\usepackage{subcaption}
\usepackage{amsfonts}
\usepackage{listings}
\usepackage{xcolor}
\usepackage{float}
\usepackage{wrapfig}

\definecolor{codegreen}{rgb}{0,0.6,0}
\definecolor{codegray}{rgb}{0.5,0.5,0.5}
\definecolor{codepurple}{rgb}{0.58,0,0.82}
\definecolor{backcolour}{rgb}{0.95,0.95,0.92}

\lstdefinestyle{mystyle}{
    backgroundcolor=\color{backcolour},   
    commentstyle=\color{codegreen},
    keywordstyle=\color{magenta},
    numberstyle=\tiny\color{codegray},
    stringstyle=\color{codepurple},
    basicstyle=\ttfamily\footnotesize,
    breakatwhitespace=false,         
    breaklines=true,                 
    captionpos=b,                    
    keepspaces=true,                 
    numbers=left,                    
    numbersep=5pt,                  
    showspaces=false,                
    showstringspaces=false,
    showtabs=false,                  
    tabsize=4
}

\lstset{style=mystyle}

\newcommand{\IFa}{\uparrow\mathrm{IF}}
\newcommand{\IFr}{\downarrow\mathrm{IF}}
\newcommand{\IDa}{\uparrow\mathrm{ID}}
\newcommand{\IDr}{\downarrow\mathrm{ID}}
\newcommand{\FUa}{\uparrow\mathrm{FU}}
\newcommand{\FUr}{\downarrow\mathrm{FU}}
\newcommand{\COM}{\mathrm{COM}}

%============================================
\usetikzlibrary{arrows,shapes,positioning,shadows,trees}
\makesavenoteenv{tabular}
\makesavenoteenv{table}
%==============================================
\def\checkmark{\tikz\fill[scale=0.4](0,.35) -- (.25,0) -- (1,.7) -- (.25,.15) -- cycle;}
%Style des têtes de section, headings, chapitre
\headstyles{komalike}
\nouppercaseheads
\chapterstyle{dash}
\makeevenhead{headings}{\sffamily\thepage}{}{\sffamily\leftmark} 
\makeoddhead{headings}{\sffamily\rightmark}{}{\sffamily\thepage}
\makeoddfoot{plain}{}{}{} % Pages chapitre. 
\makeheadrule{headings}{\textwidth}{\normalrulethickness}
%\renewcommand{\leftmark}{\thechapter ---}
\renewcommand{\chaptername}{\relax}
\renewcommand{\chaptitlefont}{ \sffamily\bfseries \LARGE}
\renewcommand{\chapnumfont}{ \sffamily\bfseries \LARGE}
\setsecnumdepth{subsection}


% Title page formatting -- do not change!
\pretitle{\HUGE\sffamily \bfseries\begin{center}} 
\posttitle{\end{center}}
\preauthor{\LARGE  \sffamily \bfseries\begin{center}}
\postauthor{\par\end{center}}
\newcommand{\jury}[1]{% 
\gdef\juryB{#1}} 
\newcommand{\juryB}{} 
\newcommand{\session}[1]{% 
\gdef\sessionB{#1}} 
\newcommand{\sessionB}{} 
\newcommand{\option}[1]{% 
\gdef\optionB{#1}} 
\newcommand{\optionB} {}

\usepackage{xcolor}
\newcommand{\TODO}[1]{\colorbox{red!30}{TODO: \textbf{#1}}}
% \newcommand{\TODO}[1]{}

\usepackage[most]{tcolorbox}
\newcounter{example}
\newtcolorbox{examplebox}[2][]{
    colback=gray!10,
    colframe=gray!80,
    boxrule=0.5pt,
    arc=3pt,
    left=6pt,
    right=6pt,
    top=4pt,
    bottom=4pt,
    fonttitle=\bfseries,
    title={Example~#2:~#1}
}
\newenvironment{example}[1][]{
    \refstepcounter{example}
    \begin{examplebox}[#1]{\theexample}
}{
    \end{examplebox}
}

\newcounter{assumption}
\newtcolorbox{assumptionbox}[2][]{
    colback=blue!5,
    colframe=blue!40,
    boxrule=0.5pt,
    arc=3pt,
    left=6pt,
    right=6pt,
    top=4pt,
    bottom=4pt,
    fonttitle=\bfseries,
    title={Assumption~#2:~#1}
}
\newenvironment{assumption}[1][]{
    \refstepcounter{assumption}
    \begin{assumptionbox}[#1]{\theassumption}
}{
    \end{assumptionbox}
}

\renewcommand{\maketitlehookd}{% 
\vfill{}  \large\par\noindent  
\begin{center}\juryB \bigskip\sessionB\end{center}
\vspace{-1.5cm}}
\renewcommand{\maketitlehooka}{% 
\vspace{-1.5cm}\noindent\includegraphics[height=12ex]{pics/logo-uga.png}\hfill\raisebox{2ex}{\includegraphics[height=14ex]{pics/logoINP.png}}\\
\bigskip
\begin{center} \large
Master of Science in Informatics at Grenoble \\
Master Informatique \\ 
Specialization \optionB  \end{center}\vfill}
% =======================End of title page formatting

\option{Parallel Computing and Distributed Systems} 
\title{Timing Anomaly through Branch Prediction} %\\\vspace{-1ex}\rule{10ex}{0.5pt} \\sub-title} 
\author{Andrei Ilin}
\date{Defense Date, 2025} % Delete this line to display the current date
\jury{
Research project performed at VERIMAG \\\medskip
Under the supervision of:\\
Lionel Rieg\\\medskip
Florian Brandner\\\medskip
Mihail Asavoae\\\medskip
% Defended before a jury composed of:\\
% Head of the jury\\
% Jury member 1\\
% Jury member 2\\
}
\session{June \hfill 2025}
\setcounter{tocdepth}{4}
\setcounter{secnumdepth}{4}

%%% BEGIN DOCUMENT
\begin{document}
\selectlanguage{English} % french si rapport en français
\frontmatter
\begin{titlingpage}
\maketitle
\end{titlingpage}

%\small
\setlength{\parskip}{-1pt plus 1pt}

\renewcommand{\abstracttextfont}{\normalfont}
\abstractintoc
\begin{abstract} 
In this report, we consider counter-intuitive timing anomalies (TAs) due to branch prediction in out-of-order (OoO) processor pipelines. A TA is a phenomenon when a local speedup in instruction execution leads to a global slowdown. We overview the existing TAs definitions and evaluate the applicability of the most promising one to branch prediction, using a novel ad-hoc model checking tool. Finally, we show that although the chosen definition adequately capture some TAs caused by branch prediction, its definition of causality cannot always capture the proper causality connections. This suggests that finding consistent TA definition encompassing branch prediction is still an open problem.

\end{abstract}
\abstractintoc

% \renewcommand\abstractname{Acknowledgement}
% \begin{abstract}
% I would like to express my sincere gratitude to .. for his invaluable assistance and comments in reviewing this report... 
% Good luck :) 
% \end{abstract}


\renewcommand\abstractname{R\'esum\'e}
\begin{abstract} \selectlanguage{French}
Dans ce rapport, nous \'etudions les anomalies temporelles contre-intuitives dues \`a la pr\'ediction de branchement dans les pipelines de processeurs \`a ex\'ecution dans le d\'esordre. Une TA est un ph\'enom\`ene o\`u une acc\'el\'eration locale de l'ex\'ecution d'une instruction conduit \`a un ralentissement global de l'ex\'ecution. Nous passons en revue les d\'efinitions existantes des TAs et \'evaluons l'applicabilit\'e de la plus prometteuse \`a la pr\'ediction de branchement, en utilisant un nouvel outil de model checking ad hoc. Enfin, nous montrons que, bien que la d\'efinition choisie capture correctement certaines TAs caus\'ees par la pr\'ediction de branchement, sa d\'efinition de la causalit\'e ne permet pas toujours de bien saisir les bons liens de causalit\'e. Cela sugg\`ere que concevoir une d\'efinition coh\'erente des TAs englobant la pr\'ediction de branchement reste un probl\`eme ouvert.
\end{abstract}
\selectlanguage{English}

\cleardoublepage

\tableofcontents* % the asterisk means that the table of contents itself isn't put into the ToC
\normalsize

\mainmatter
\SingleSpace
%==============================CHAPTERS==================
\chapter{Introduction}

\TODO{what is a critical system, how it is different}

In critical systems such as airplanes and cars, it is important that tasks running on the hardware meet their deadlines. For example, in a car's braking system, missing a program deadline can lead to loss of control over the vehicle. Therefore, it is crucial to have a rigorous analysis that can provide an upper bound for the program's execution time on a given hardware platform. In non-critical real-time applications, the execution time bound can be estimated by measuring the execution time for many input values. This gives the maximum observed execution time, which usually underestimates the real \textit{worst-case execution time (WCET)}.

In critical real-time systems, this approach is not enough because some cases may be missed during observation, as shown in Figure \ref{fig:timing-distribution}. The real WCET can only be found by testing the program on all possible inputs, which is usually not possible due to the large number of cases. Therefore, methods to estimate WCET bounds use abstractions of the program and hardware. These methods may overestimate the WCET because of simplifications, but they are more scalable.

\begin{figure}[H]
    \centering
    \includegraphics[width=\textwidth]{figures/timing-distribution.png}
    \caption{Timing analysis notations. The lower curve shows a subset of measured executions. The darker curve, an envelope of the former, shows the times of all executions (from \cite{ferdinand_worst_2004}).}
    \label{fig:timing-distribution}
\end{figure}

Usually, WCET analysis is divided into several phases, each focusing on a part of the hardware or software. For example, in the software part, memory analysis assigns address bounds for each instruction \cite{Harrison_Ranges_1977}, and loop bounds analysis finds the bounds for loop iterations as constants or formulas \cite{Healy_bounding_1998}. On the hardware side, there are analyzes for cache, main memory access latency, and pipeline. Together, these analyzes help to find the longest path of a program, which gives the WCET bound.

\begin{figure}
    \centering
    \includegraphics[width=0.7\textwidth]{figures/wcet-deps.png}
    \caption{WCET analysis phases and their dependencies. (\TODO{Image source})}
    \label{fig:wcet-deps}
\end{figure}

Dividing WCET analysis into phases creates the phase ordering problem: one phase may need information from another. Sometimes, these dependencies are circular. For example, in Out-of-Order (OoO) processors, the instruction order depends on the architecture state, which also affects cache access. This means cache analysis needs pipeline analysis. For timing analysis, it is desirable to have composable systems \cite{Puschner_computing_1997}.

However, most architectures are not composable and have so-called \textit{timing anomalies (TA)}. A TA happens when local worst cases do not lead to a global worst case. TAs are seen in pairs of execution traces where the initial hardware state is different, but the instruction sequences are the same. Different cache states can cause variations in timing because of a miss in one trace and a hit in another.

\begin{example}
Figure \ref{fig:TA1} shows an example of such an anomaly. The assembly sequence has 4 instructions ($A,B,C,D$) with data dependencies $A \rightarrow B$ and $C \rightarrow D$. Figure \ref{fig:TA1-trace} shows two traces ($\alpha, \beta$) from running the program. There is a difference in the latency of instruction $A$ ($1$ in $\alpha$ and $2$ in $\beta$). In trace $\alpha$, the variation seems better, but the total execution time is higher, which shows an anomaly.
\label{ex:simple-ta}
\end{example}

\begin{figure}[H]
    \centering
    \begin{subfigure}[t]{0.3\textwidth}
        \centering
        \includegraphics[width=\textwidth]{figures/first-TA-ex-input.png}
        \caption{Input assembly sequence}
        \label{fig:TA1-code}
    \end{subfigure}
    \hfill
    \begin{subfigure}[t]{0.5\textwidth}
        \centering
        \includegraphics[width=\textwidth]{figures/first-TA-ex-trace.png}
        \caption{Scheduling on functional units comparison}
        \label{fig:TA1-trace}
    \end{subfigure}
    \caption{TA caused by variation in latency of instruction \textit{A} (from \cite{binder_definitions_2022})}
    \label{fig:TA1}
\end{figure}

Timing anomalies are a serious problem for timing predictability. Understanding their nature is important to estimate their impact on global execution time or to design hardware without TAs. Many hardware features can cause TAs, such as caches and memory buses. In this work, we focus on branch prediction. We start by explaining the microarchitecture context, then review existing definitions, and finally present our framework to find branch-related TAs and propose a formal definition for them.

\chapter{Background}

\section{Instruction Set architecture}

Instruction Set Architecture (ISA) defines the set of instructions and the registers on which they operate. Normally, the instruction operands are read from the registers and the execution result is stored there. ISA serves as an interface between software and actual hardware microarchitecture which implements the ISA. 

\TODO{what is ISA-state, instructions-granularity. ISA-level reigsters, mapping to real regs}

\section{Microarchitecture}

ISA defines binary format of instructions which are stored in memory and accessed by the processor through cache mechanisms, usually, fixed length instructions are used, while мariable-length also exist. (\TODO{add examples}). Processor is a cycled device that performs fetching instructions from memory and their subsequent execution, we call the microarchitectural state the state of all hardware registers of the processor. Unlike in ISA, states are defined at clock-cycle granularity, so an instruction takes several clock-cycles to finish. Different optimizations, such as pipelining, multiscalar execution, out-of-order execution and branch predictors (speculative execution).

\subsection{Processor Pipeline Stages}

Each instruction needs several microopererations to be executed: first, the instruction is to be loaded from memory, the operands need to be loaded from the registry. After that the instruction is executed during several cycles depending on its type (for example, multiplication is longer than addition). Due to the fact of isolation of those micropererations, it is possible to execute several instructions simultaneously: when instructions free its stage, next instruction enters it. This optimization, called pipelining, allows to increase the throughput of the processor.

\TODO{Is the word "microopererations" correct here?}

Several decompositions can exist for modern processors. Here we describe the 5 stages that can found in any processor and some of which may be further decomposed in more sophisticated architectures.

\subsubsection{Instruction Fetch (IF)}

As it was said before, the program instructions reside in global memory. This means that instructions access needs to be performed through memory hierarchy using program counter (PC) address. Often, a special instruction cache exists for accessing the program. IF stage is also responsible for updating PC to read the new instruction.

\subsubsection{Instruction Decode (ID)}

Once the instruction is fetched from memory, it exists in a processor in a packed binary format. This encoding includes the type of instruction as well as the registers it operates with. Decode stage loads the actual values from Physical Registry File (PRF) and propagates them to downstream pipeline stages. Sometimes the value can be obtained through bypass network before it appears in PRF.

\subsubsection{Execute (EX)}

EX stage computes the result of the operation. Several components may be responsible for performing different types of operations (for instance, different components for addition and multiplication). In this case IF stage emits control signals that determine the data path.

In case of memory or jump instruction the address is calculated.

The result of the computation is directly available to the ID stage via bypass network.

\subsubsection{Access Memory (MEM)}

This stage performs access to the global memory through memory hierarchy. If instruction is not a memory instruction, this stage is skipped.

\subsubsection{Commit (COM)}

The purpose of the last stage is to write the result of the instruction to PRF. Only after this the result is visible from ISA-state perspective.

\subsection{Restrictions}

The structure of the program imposes limitations on execution. Instruction may block each other thus stalling the pipeline. 

\subsubsection{Data Hazards}

There exist three types of register dependencies that may cause pipeline stall.

\textbf{Read-After-Write (RAW)} dependencies, also called as true data dependencies, arise when to perform one operation, the result of the other must be obtained. For example expression $(1 + 2 * 3)$ requires $(2 * 3)$ be calculated first, thus creating \textit{RAW}-dependency between multiplication and addition operations.

\textbf{Write-After-Write (WAW)} dependency happens when two instructions are writing to the same ISA-level register. The two writes must happen is instruction order.

\textbf{Write-after-Read (WAR)} dependency exists when the younger instruction aims at writing a value in the register which is to be read by an older instruction.

RAW-hazards are inevitable in any architecture. WAW and WAR dependencies do not exist in the model we described so far, by must be resolved in out-of-order pipeline.

\subsubsection{Control Hazards}

Fetching next instruction is possible only if the address of it is known. In case of branches, the next instruction address is not known until the branch outcome is calculated in execute stage.

Therefore, the so-called bubbles (which denote the absence of operation) are introduced into the pipeline.

\begin{figure}
    \includegraphics[width=\textwidth]{figures/pipeline-bubbles.png}
    \caption{Example of control hazard: the pipeline is stalled until branch finished the execution (from \cite{perais_increasing_2016})}
    \label{fig:bubbles}
\end{figure}

\subsection{Multiscalar Execution}

Instead of fetching instructions one by one, it is possible to fetch several ones in the same time. This also means that other stages are also multiplied to accommodate all fetched instructions. Since neighbor instructions may be independent this can significantly increase the performance. However, duplicating each stage is costly, while it is relatively easy for IF, ID and COM, execution and accessing memory is much harder to duplicate.

\subsection{Out-of-Order (OoO) Pipeline}

Despite the fact that the instructions are to be processed in program order, many of them are in fact independent. This means that the order of execution can be chosen based on instruction dependencies rather than their order in initial program. Notice that the pipeline is often stalled by the execution of long instructions (\TODO{refer visual example}). The key idea is that while one instruction is being executed on one functional unit (FU), the other, independent of this one can be executed on the other FU. 

In this approach we divide the pipeline into in-order and out-of-order parts. In-order consists of IF, ID and COM stages while out-of-order includes execution and memory accesses. This allows to achieve a consistent ISA-stage due to in-order fetch a commit. 

Different mechanisms exist to synchronize out-of-order execution. Here we introduce reservation stations (RS) and reorder buffer (ROB) - the additional pipeline stages.

Reservation station is a queue before the functional unit, each FU is equipped with its own RS. Once the instruction is decoded it is forwarded to FU based on its type, but if FU is busy, the instruction is put instead into the corresponding RS. Subsequently, the FU is taking the instructions both from ID and RS based on the scheduling policy.

ROB is a FIFO queue that insures the order in which instructions should be committed. Each time, the instruction enters out-of-order part (RS or FU) it is also appended to the front of ROB. After being executed, the instruction is tagged as ready in the ROB. The COM stage commits only the last instruction (or several if multiscalar) from the ROB if it is ready, thus ensuring commit in program order.

\TODO{Image}

\subsection{Branch Prediction}

IF stage is responsible for fetching the next instruction in the program. However, when conditional jump instruction is fetched the next read address is undefined until the outcome of condition is calculated. The straightforward approach is to stall the pipeline, introducing so-called bubbles (no operation).

The more advanced approach consists of fetching a new instruction anyway, the address of which is guessed by branch prediction mechanism, discussed further. Such instructions are called speculative and are not committed until branch decision is taken. In case of incorrect prediction speculative instruction are flushed from the pipeline. 

\section{Branch Predictor Implementations}

\subsection{Static Branch Predictors}

Static branch prediction relies on information known at compile time. Some well-known static branch predictors are:

\begin{itemize}
    \item Always Not Taken
    \item Always Taken
    \item Backward Taken, Forward Not Taken
\end{itemize}

\TODO{add details}

\subsection{Dynamic Branch Predictors}

Dynamic Branch Predictors rely on information retrieved from execution and are usually based on previous branch outcomes. The usage of dynamic branch predictors requires additional hardware components which are discussed below.

\textbf{Pattern History Table (PHT)} is used to store information about each branch. It can be a bit denoting whether the branch was taken last time, or a more complex data. PHT is usually indexed by the lower bits of branch instruction address.

\textbf{Branch Target Buffer (BTB)} stores the destinations of previously computed branch. When starting speculative execution, values from BTB are used.

\textbf{Return Stack Buffer (RSB)} is used to predict the outcome of \textit{ret} instructions.

\subsubsection{One-Bit Predictor}

The one-bit predictor is the simplest type of dynamic branch predictor. It uses PHT indexed by lower bits of address where one-bit value encodes the last branch outcome. Such a simple predictor is efficient when branch decision is not often changed throughout execution. For example, loop conditions are mispredicted only twice by this type of predictor: on the first and the last iterations of the loop.

However, more complex patterns diminish the efficiency of one-bit predictor. For instance, if branch outcome changes each time, the predictor accuracy is zero.

\subsubsection{Two-Bit Predictor}

The two-bit predictor uses the same idea of PHT-indexing, but instead of storing just the outcome of previous branch, it has 4-state automaton encoded by 2 bits. The states are STRONG-TAKEN, WEAK-TAKEN, WEAK-NTAKEN and STRONG-NTAKEN. Picture \TODO{} shows the transitions between the states.

\begin{figure}
    \includegraphics[width=\textwidth]{figures/two-bit-counter.png}
    \label{fig:two-bit-counter}
    \caption{Two-bit predictor state machine (from \cite{mahling_reverse_2023})}
\end{figure}

\TODO{why better than 1-bit}

\TODO{other types. which are used in critical systems?}


\section{WCET Analysis}

In critical systems such as \TODO{examples} it is important that the tasks executed on the hardware meet their deadlines. This is ensured by worst execution time (WCET) analysis. It takes the pair of the program and the dedicated hardware and aims at giving an upper-bound on execution time. 

\TODO{stages of WCET-analysis}

\section{Timing Anomalies}

Phase ordering is a major chellange in WCET-analysis. Most of analysis steps require information from each other (\TODO{examples}), so it is not always possible to order them. 

Nevertheless, most architectures are not composable and contain so-called timing anomalies (TA). Intuitively, TA happens when local worst cases do not constitute a global worst case. TA is observed on the pair of execution traces where the initial hardware state differs, and the instruction sequences are identical. Different cache states can be the source of variation in timing behavior due to miss in one trace and hit in another one.

\begin{example}
Figure \ref{fig:TA1} shows the example of such an anomaly. Here, the assembly sequence consists of 4 instructions ($A,B,C,D$) with data dependencies $A \rightarrow B$ and $C \rightarrow D$. Figure \ref{fig:TA1-trace} represents the pair of traces ($\alpha, \beta$) derived from execution of the given program. There is a variation in latency of instruction $A$ ($1$ in $\alpha$ and $2$ in $\beta$). In trace $\alpha$  the variation is favorable, but the total execution time is also higher in this trace which signals an anomaly.

\label{ex:simple-ta}
\end{example}

\begin{figure}[htbp]
    \centering
    \begin{subfigure}[t]{0.3\textwidth}
        \centering
        \includegraphics[width=\textwidth]{figures/first-TA-ex-input.png}
        \caption{Inpus assembly sequence}
        \label{fig:TA1-code}
    \end{subfigure}
    \hfill
    \begin{subfigure}[t]{0.55\textwidth}
        \centering
        \includegraphics[width=\textwidth]{figures/first-TA-ex-trace.png}
        \caption{Scheduling on functional units comparison}
        \label{fig:TA1-trace}
    \end{subfigure}
    \caption{TA caused by variation in latency of instruction \textit{A} (from \cite{binder_definitions_2022})}
    \label{fig:TA1}
\end{figure}

non-composable architectures

amplification and counter-intuitive TAs

\section{Execution diagrams}

\begin{figure}[htbp]
    \centering
    \includegraphics[width=0.8\textwidth]{figures/multiscalar_ta.png}
    \caption{Execution traces from example \ref{ex:simple-ta}}
    \label{fig:multiscalar-ta}
\end{figure}

\TODO{what is trace, what is variation}

\TODO{vertical diagram}

\TODO{diagonal diagram}

\TODO{execution trace vs instruction trace distinction}


\chapter{Timing Analysis and Anomalies}

\section{Evolution of TA-definitions}

A timing anomaly (TA) is a situation where a local favorable condition leads to a globally worse state (for example, a cache hit leading to slowdown of the program).  The notion of timing anomalies dates back to 1999 when they were first introduced by Lundqvist and Stenstr\"om \cite{lundqvist_timing_1999} in context of timing analysis. Basically, TA is a feature of an architecture which makes it hard to analyze timing behavior properly. Such anomalies may have a tremendous impact on execution time which is not captured by the WET analysis. Especially dangerous is so-called domino effect, also discovered by Lundqvist and Stenstr\"om. It leads to an unbounded slowdown effect of the TA.

Despite the fact that timing anomalies have been known for a long time, the exact TA definition is a subject to debates. Since 1999 several attempts were made to formalize the notion of TA, some of them being more focused on the exact microarchitecture (like \cite{gruin_minotaur_2023}) and some being more abstract and general (like \cite{binder_definitions_2022}, \cite{hahn_design_2020}). In this section we are giving an overview of existing definitions comparing their strengths and weaknesses.

\subsection{Step Heights}

Gebhard \cite{gebhard_timing_2012} gives a timing-anomaly definition based on local execution time of instructions in comparison to global execution time defined as sum of local ones. A TA exists when local execution time of earlier instruction is lower and the global execution time of some later instruction is higher (compared to other trace).

Figure \ref{fig:step-good} shows this definition applied to example \ref{ex:simple-ta}. Orange arrow illustrates the local execution time of instruction $A$. The global time for instruction $D$ is different between traces $\alpha$ and $\beta$ (13 and 11 respectively).

\begin{figure}[!htb]
    \centering
    \begin{subfigure}[t]{0.5\textwidth}
        \centering
        \includegraphics[width=\textwidth]{figures/step-func-good.png}
        \caption{Interpretation of example \ref{ex:simple-ta} using Gebhard's definition}
        \label{fig:step-good}
    \end{subfigure}
    \hfill
    \begin{subfigure}[t]{0.49\textwidth}
        \centering
        \includegraphics[width=\textwidth]{figures/step-func-bad.png}
        \caption{Counterexample to the definition}
        \label{fig:step-bad}
    \end{subfigure}
    \caption{Gebhard's definition applied to execution traces (from \cite{binder_definitions_2022})}
    \label{fig:step}
\end{figure}

In his thesis \cite{binder_definitions_2022}, Binder provides a counterexample (figure \ref{fig:step-bad}), where it is clear that there is no TA (trace $\beta$ has both unfavorable variation and longer execution time). However, Gebhard's definition signals an anomaly because of shorter local execution time of instruction $C$ in trace $\beta$.

This poses a question whether it is reasonable to capture a local execution time as difference between instruction completion times. 

\subsection{Step-functions Intersections}

Similar definition is proposed by Cassez et al. \cite{cassez_what_2012}. The difference is that only global execution time is taken into account. Thus, TA arises when step-functions (that map instructions to their absolute completion time) of two traces intersect. 

\begin{figure}[!htb]
    \centering
    \includegraphics[width=\textwidth]{figures/step-func-2-bad.png}
    \caption{Contradicting result of Cassez's definition (from \cite{binder_definitions_2022})}
    \label{fig:step-2}
\end{figure}

This definition also leads to misleading effect with scenario found by Binder \cite{binder_definitions_2022}. Figure \ref{fig:step-2} illustrates this by comparing two traces. Step-functions of traces $\alpha$ and $\beta$ intersect, however there is no counter-intuitive TA happening as $\alpha$ is both longer and has a  longer latency for IF stage of instructions $C$ and $D$.

\subsection{Component Occupation}

An alternative approach is proposed by Kirner et al. \cite{kirner_precise_2009}. In their work the idea is to partition hardware into components and for each define the occupation by instruction (for how many cycles it processes the instruction). TA arises when a shorter component occupation coincides with a longer execution time in a chosen trace. However, as is shown is \cite{binder_definitions_2022} the results depend on how we define component partition which imposes the major concern against using this definition.

\TODO{counterexample}

\subsection{Instruction Locality}
\subsection{Progress-based definition}

Hahn and Reineke \cite{hahn_design_2020} introduce the notion of progress, ... \cite{gruin_minotaur_2023}

\subsection{Event Time Dependency Graph}

Binder et al. \cite{binder_definitions_2022} define TAs using the notion of causality between events in execution trace. In this work, multiscalar OoO pipeline is considered. The processor state is described as a composition of states of each of the resource: \textit{IF, ID, set of RS, set of FU, ROB, COM}. Each component holds the information about instruction it is currently processing, including required registers and remaining clock cycles.

Notion of event is introduced based on qualitative changes in the pipeline associated to instruction progressing through stages. Event from execution trace (denoted as $e \in Events(\alpha$)) is a triple $(i,r,t)$, where $i$ is the instruction to which event is related, $r$ is the associated resource and the action (acquisition or release) and $t$ is a timestamp corresponding to the clock cycle when event occurs.

In the proposed framework events are related to \textit{IF, ID, FU} and \textit{COM} stages. For each instruction there are 7 types of events: $\IFa$, $\IFr$, $\IDa$, $\IDr$, $\FUa$, $\FUr$ and $COM$. $\uparrow$ signs the acquisition of a resource and $\downarrow$ its release. $COM$ denotes the acquisition of the commit stage; hence its release always happens one clock cycle after and no subsequent stages exist, it is not included into framework.

\textbf{Latency} is defined as the time difference between the acquisition and release of a resource. Each instruction passes through the same pipeline stages and is associated with corresponding events. Therefore, for each pair of traces corresponding to the same program, the sets of events differ only in their timestamps or, potentially, in the functional unit (FU) used (although resource switching is not modeled within this framework). Consequently, for each event in one trace, there exists a corresponding event in the other. Formally, this correspondence is defined by the function $CospEvent: Events(\alpha) \rightarrow Events(\beta)$.


A \textbf{variation} signs that the latency in one trace differs from latency of corresponding events in the other trace. On the pair of traces $\alpha$ and $\beta$. The variation is considered favorable for $\alpha$ if the latency in $\alpha$ is smaller than in $\beta$.

Variations are chosen as a source of timing anomalies. They may represent different memory behavior (cache hit or miss) for fetch and memory accesses in FU. Other sources of TA such as memory bus contention or branching are not considered by the framework.

\textbf{Event Time Dependency Graph (ETDG)} of trace $\tau$ denoted as $G(\tau) = (\mathcal{N}, \mathcal{A})$ is composed of a set of nodes $\mathcal{N} = Events(\tau)$ and a set of arcs $\mathcal{A} \subseteq \mathcal{N} \times \mathcal{N} \times \mathbb{N}$. 


Arc is a triple $(e_1, e_2, w)$ written as $e_1 \xrightarrow{w} e_2$ where $e_1$ is the source event node, $e_2$ -- destination node and $w$ is a lower bound of the delay between the two events. The arc means that at least $w$ clock cycles must pass between $e_1$ and $e_2$. 

Arcs are derived from a set of rules:
\begin{enumerate}
    \item \textbf{Order of pipeline stages}
    
    $(I, \IFa, t_0) \xrightarrow{lat_{IF}} (I, \IFr, t_1) \xrightarrow{0} (I, \IDa, t_2) \xrightarrow{1} (I, \IDr, t_3) \xrightarrow{0} (I, \FUa, t_4)  \xrightarrow{lat_{FU}} (I, \FUr, t_5)  \xrightarrow{0} (I, COM, t_6)$

    $lat_{IF}$ and $lat_{FU}$ are the latencies of IF and FU stages respectively.

    \item \textbf{Resource use}
    
    $lat_{IF} = t_1 - t_0$, $lat_{FU} = t_5 - t_4$

    \item \textbf{Instruction order}
    
    In-order part of the pipeline is constrained by instruction order. Thus, for successive instructions $I_1$ and $I_2$:

    $(I_1, RES\uparrow, t) \xrightarrow{0} (I_2, RES\uparrow, t'), RES \in \{IF, ID, COM\}$


    \item \textbf{Data dependencies}
    
    RAW dependency between $I_1$ and $I_2$ (\TODO{dep notation}) restricts the execution order of the instructions:  $(I_1, \FUr, t) \xrightarrow{0} (I_2, \FUa, t')$.
    
    \item \textbf{Resource contention}
    
    Also some instruction can be delayed because of limited resources. For instance, FU contention happens when $I_1$ and $I_2$ use the same FU, and it is busy by $I_1$ at the moment when $I_2$ is ready. This creates $(I_1, \FUr, t) \xrightarrow{0} (I_2, \FUa, t')$. 

    Resource contention can also be caused by reaching the capacity limit of ROB or RS. 
\end{enumerate}

\textbf{Causality graph} is achieved from ETDG by removing unnecessary edges. For each event we keep only the most relevant constraint. Only arcs of the form $e_1 \xrightarrow{e_2.time - e_1.time} e_2$ are left. Also arcs related to variations are excluded.

\textbf{Timing anomaly} is observed on pair of traces $\alpha$ and $\beta$ if there exists a favorable variation in $\alpha$ relative to $\beta$. Let $e_\alpha\downarrow$ and $e_\beta\downarrow$ be the events corresponding to the end of the variation in both traces. If there exist events $e_\alpha$ and $e_\beta$, where $e_\beta = CospEvent(e_\alpha)$ and there is a path in causality graph of $\alpha$ between $e_\alpha\downarrow$ and $e_\alpha$, s.t. $\Delta(e_\beta\downarrow,e_\beta) < \Delta(e_\alpha\downarrow,e_\alpha)$.


\begin{figure}[htbp]
    \centering
    \includegraphics[width=0.8\textwidth]{figures/multiscalar_ta_causality.png}
    \caption{Causality-based TA detection applied to Example \ref{ex:simple-ta}. $e_\alpha\downarrow = (A, \FUr, 4), e_\beta\downarrow = (A, \FUr, 6), e_\alpha = (A, COM, 13), e_\beta = (A, COM, 11)$. Purple arrow denotes latency which has a variation between two traces. Gray arrow shows delay between events which is greater in favorable trace. Causality in  path $\alpha$ is marked by red arrows.}
    \label{fig:multiscalar-ta-causality}
\end{figure}

\begin{figure}[htbp]
    \centering
    \includegraphics[width=\textwidth]{figures/ETDG.png}
    \caption{Complete ETDG for trace $\alpha$ from figure \ref{fig:multiscalar-ta}. \TODO{image source}}
    \label{fig:ETDG}
\end{figure}

Figure \ref{fig:multiscalar-ta-causality} shows how the framework captures TA for example \ref{ex:simple-ta}. Figure \ref{fig:ETDG} presents the complete ETDG for trace $\alpha$ with different dependency rules highlighted with different colors. The arcs reflecting causality are depicted in solid lines.

In contrast to other definition, this one measures relative time from the acquisition of the resource instead of global time. This approach allows the separation of different variations and isolates the part of the trace that experiences TA-effect.


\section{TA-classifications}


\chapter{Contribution}

Definition of Binder et al. \cite{binder_definitions_2022} is promising, however the branch prediction and the related issues are not taken in account by the framework. Thus we aim to extend the use case of proposed definition.

In our work we try to adjust Binder's definition to the setting of pipeline with branch predictor. We introduce an input format capable of expressing speculative execution. 

\TODO{complete intro when chapter is done}

\section{Methodology}
To systematically investigate timing anomalies induced by branch predictors, it is essential to efficiently generate and analyze relevant examples. Manual construction of such examples is both time-consuming and error-prone, motivating the need for a tool that can automatically or semi-automatically produce and validate them. With such a tool, we can iteratively generate candidate scenarios, analyze their behavior, and assess the applicability of Binder et al.'s definition to these new cases. This process enables us to refine and adapt the definition as necessary, guided by empirical evidence from the generated examples.

Our initial efforts focused on studying and extending the TLA$^+$ \cite{lamport_specifying_2003} framework developed by Binder et al.  to support branch behavior. However, we encountered significant performance limitations and found the input format insufficiently flexible for rapid prototyping and adjustment of examples.

Consequently, we reimplemented the framework in C++, drawing on insights from the TLA$^+$ model. We also use TLA$^+$ as a reference to check the correctness. The new implementation offers substantial performance improvements and introduces randomized search capabilities, enabling efficient exploration of the space of possible instruction traces and facilitating the discovery of timing anomalies related to branch prediction.

\section{Framework}

\subsection{Existing Framework Overview}

\subsubsection{Exploration by Model Checking}

The implementation provided by Binder is written in TLA$^+$ \cite{lamport_specifying_2003}. The pipeline state is specified in set-theory notation. The model checker step corresponds to a one clock cycle and derives a new HW state from the previous one. This allows to simulate the non-deterministic timing behavior: each time when a variation can happen, multiple next state are generated. TLA$^+$ covers all reachable states ensuring that all possible behaviors are covered.

The pair of trace constitutes a whole model state. TA is expressed as an invariant for the pair of traces, so its is verified in each model checking step. 

As well as a construction of traces, the framework provides visualization methods for the traces and ETDG.

% \TODO{each pair of executions is considered? or all executions are compared against the one reference?}

\subsubsection{Input Trace Format}

The input of the framework is a pair of:
\begin{enumerate}
	\item Pipeline parameters: superscalar degree, $FU$ latencies and memory access latencies depending on the cache events (hit or miss). sequence of instructions;
	\item Instruction sequence: for each instruction its type and registers are specified as well as set of cache behaviors to be explored by the model checker. The type is used to know which $FU$ will be used by the instruction and based on registers data dependencies are retrieved.
\end{enumerate}

Figure \ref{fig:TLA-format} illustrates the instruction sequence that causes TA in Example \ref{ex:simple-ta}. We can simplify this view by directly expressing the resource, dependencies and possible latencies of instruction. Figure \ref{fig:input-format} shows the input for instruction trace from example \ref{ex:simple-ta}. First column is instruction label, second is the resource used, thirds is the set of data dependencies and the last one captures possible execution latencies. In the same fashion we could specify variations of latencies for $IF$ stage, but we skip them for simplicity. This table is sufficient to express a pair of execution traces derived from instruction trace.

\begin{figure}[H]
\begin{lstlisting}[basicstyle=\fontsize{8}{13}\selectfont\ttfamily]
missLat == 3
mayDMiss == {1}
program == <<
[ ind |-> 1, type |-> "MemRead",  r0 |-> "ra", r1 |-> "",   r2 |-> "", addr |-> "0x1" ],
[ ind |-> 2, type |-> "IntAlu",   r0 |-> "",   r1 |-> "ra", r2 |-> "", addr |-> "0x2" ],
[ ind |-> 3, type |-> "IntAlu",   r0 |-> "rb", r1 |-> "",   r2 |-> "", addr |-> "0x3" ],
[ ind |-> 4, type |-> "MemWrite", r0 |-> "",   r1 |-> "rb", r2 |-> "", addr |-> "0x4" ]
>>
\end{lstlisting}
\caption{TLA$^+$ input format for Binder's framework. Some lines are excluded for brevity}
\label{fig:TLA-format}
\end{figure}


\begin{figure}[htbp]
	\centering
	\begin{tabular}{r|ccc}
    & Resource & Dependencies & Latencies \\ \hline
    \textit{A:} & FU1 &  & $1 | 3$ \\
    \textit{B:} & FU2 & $\{A\}$ & $3$ \\
    \textit{C:} & FU2 &  & $3$ \\
    \textit{D:} & FU1 & $\{C\}$ & $3$ \\
    \end{tabular}

	\caption{Simplified input format of example from figure \ref{fig:TA1-code}}
	\label{fig:input-format}
\end{figure}

\subsection{Limitations}

Despite using a model checker, the existing framework is capable to explore only the traces that fit the instruction template. This limits the explored space to what is manually defined by the user. Considering that branches are to be added, this limitation is becoming even more restricting. 

Nevertheless, the framework may be used to manually specify the instruction trace using a template and generate a resulting pair execution traces. This allows to quickly sketch the examples and analyze them. Unfortunately this feature comes up with some issues.

The significant flaw we noticed was the performance. Firstly, the TLA$^+$ itself takes a few seconds to generate initial states of the model. Secondly, the graph is analyzed using java embedding which calls a script in python which in its turn deserializes a graph from text output of the model checker tool. 

Moreover, the input is specified in lengthy TLA$+$ notation, which prevents fast sketching the examples. Thus it was decided not to write an extension of the existing framework, but to design a new one from scratch.

\subsection{Our Novel Framework}

We introduce a novel framework inspired by Binder et al., designed to address the limitations of the original implementation. Our framework features a lightweight input format that natively supports branch behavior and speculative execution, enabling concise and intuitive specification of instruction traces. To overcome the performance bottlenecks of TLA$^+$, we implement our solution in C++, providing significant speedup and enabling real-time feedback for rapid prototyping. Also, the performance enhancement allows to explore the larger state spaces effectively. Our framework facilitates efficient analysis of timing anomalies and supports both manual and automated exploration modes.

\subsubsection{Misprediction Region}

The format of the input traces was adapted to handle speculative execution. We decided to use a simplified format as in figure \ref{fig:input-format} as a baseline. In Binder's framework instruction trace format is straightforward: it specifies all instructions that are fetched, executed and finally committed. In case of speculative execution, some instructions enter the pipeline, but are never committed, being squashed by the resolution of the branch. To tackle this problem we introduce the notion of \textit{misprediction region of branch instruction}.

As an input trace we specify all instructions that can enter the processor pipeline. As we focus only on timing behavior of the program, abstracting from memory and registers state, we also assume that the control flow is known for a given instruction trace. Thus for each branch we may specify the instructions in only one branch in case of correct prediction. However, in case of misprediction, the instructions from the incorrect branch are fetched until the branch is resolved. We call such instructions \textit{mispredicted} and the set of such instructions after the branch a \textit{misprediction region}. 

In our input format, each line describes a single instruction, beginning with the functional unit to be used (\texttt{FU1}, \texttt{FU2}, etc.). This may be followed by an optional label, prefixed with \texttt{\#}. Data dependencies can be specified by listing the labels of dependent instructions, each prefixed with \texttt{@}. Next, the possible execution latencies are provided as a list. For branch instructions, an optional \texttt{*} denotes variation in branch prediction behavior. Misprediction regions are indicated by indentation: an indented instruction belongs to the misprediction region of the most recent less-indented branch instruction. 

Figures \ref{fig:spec-input-ugly} and \ref{fig:spec-input-pretty} present an example of the input format. Figure \ref{fig:spec-input-ugly} displays the raw input as understood by the framework, while Figure \ref{fig:spec-input-pretty} provides a more readable, tabular representation that will be used throughout the remainder of this article. In this example, instructions $C$ and $D$ reside within the misprediction region of instruction $B$. Figure \ref{fig:mispred-intro} illustrates the two possible execution traces derived from this instruction trace: trace $\alpha$ corresponds to correct branch prediction, where $C$ and $D$ are skipped and never enter the pipeline; trace $\beta$ demonstrates the misprediction scenario, in which $C$ and $D$ are fetched but subsequently squashed from the pipeline at clock cycle 5.

\begin{figure}[H]
    \centering
    \begin{subfigure}[b]{0.45\textwidth}
        \centering
\begin{lstlisting}
FU1	 #1	[4]
FU2	 @1	[1] *
    FU2 [4]
    FU2 [4]
FU2		[4]
\end{lstlisting}
        \caption{Input format understandable for framework}
        \label{fig:spec-input-ugly}
    \end{subfigure}
    \hfill
    \begin{subfigure}[b]{0.45\textwidth}
        \centering
\begin{tabular}{rr|ccc}
 &  & Res & Dep. & Lat. \\ \hline
\textit{A:} &  & FU1 &  & $4$ \\
\textit{*B:} &  & FU2 & $\{A\}$ & $1$ \\
& \textit{C:} & FU2 &  & $4$ \\
& \textit{D:} & FU2 &  & $4$ \\
\textit{E:} &  & FU2 &  & $4$ \\
\end{tabular}
        \caption{Input format used further in the text}
        \label{fig:spec-input-pretty}
    \end{subfigure}
    \caption{Two equivalent representations of input format supporting speculative execution}
    \label{fig:TA1}
\end{figure}



\begin{figure}[H]
	\centering
	\includegraphics[width=0.8\textwidth]{figures/mispred-intro.png}
	\caption{Pair of traces with correct and incorrect predictions. The squashing event is denoted with a red cross.}
	\label{fig:mispred-intro}
\end{figure}

\TODO{nested mispred region}

\TODO{,ispred region should be sufficiently large}

\subsubsection{Framework Implementation}

We decided to take C++ \cite{stroustrup_c_2015} as an implementation language as it is fast and includes a number of useful data structures in a standard library.


We define a single instruction as follows. It consists of type of FU to be scheduled at (we do not consider resource switch), latency in this FU, set of RAW dependencies. If instruction is a branch, \texttt{mispred\_region} is set to a positive value $n$ denoting the next $n$ instructions are in misprediction region of current instruction. If we want to model only the correct prediction, then \texttt{mispred\_region} is set to 0; this way branch behaves as an ordinary instruction. \texttt{br\_pred} flag specifies if the prediction is correct, which is needed when generating a pair of traces with variation in branch behavior.

\begin{lstlisting}[language=C]
struct Instr {
    int 			fu_type = 0;
    int 			lat_fu = 1;
    std::set<int> 	data_deps;
    int 			mispred_region = 0;
    bool 			br_pred = false;
};
\end{lstlisting}

At the core of our framework is the \texttt{PipelineState} structure, which models the state of all pipeline stages. The \texttt{executed} set tracks instructions that have completed execution in the functional units, enabling dependency resolution. The \texttt{branch\_stack} maintains the context for misprediction regions: each time a branch is fetched, it is pushed onto the stack and remains there until resolved. Together with the \texttt{squashed} set, this mechanism ensures correct handling of mispredicted regions. For simplicity, we do not impose capacity limits on the reservation stations (RS) or reorder buffer (ROB). The \texttt{next()} function advances the pipeline state by one clock cycle and returns whether execution has completed. It operates on the instruction sequence, which is accessed via the program counter (\texttt{pc}).

To obtain an execution trace from a given instruction sequence, we initialize an empty pipeline state (with no instructions present) and repeatedly call \texttt{next()} until the final state is reached. This process yields a sequence of pipeline states, which together form an execution trace.

\begin{lstlisting}[language=C]
struct StageEntry {
    int idx = -1;
    int cycles_left = 0;
};

struct PipelineState {
    int clock_cycle = 0;
    int pc = 0;
    vector<StageEntry> 	stage_IF = 	vector<StageEntry>(SUPERSCALAR);
    vector<int> 		stage_ID = 	vector<int>(SUPERSCALAR);
    vector<set<int>> 	stage_RS = 	vector<set<int>>(FU_NUM);
    vector<StageEntry> 	stage_FU = 	vector<StageEntry>(FU_NUM);
    vector<int> 		stage_COM = vector<int>(SUPERSCALAR);
    deque<int> 			ROB = 		deque<int>();
    set<int> 	executed;
    set<int> 	squashed;
    vector<int> branch_stack;

    bool next(const vector<Instr>& prog);
};
\end{lstlisting}

To enable efficient exploration of trace pairs that demonstrate timing anomalies (TA) and to support analysis over larger state spaces, not limited to a fixed instruction trace template, the framework provides three operating modes:

\begin{enumerate}
	\item \textbf{Manual mode}: The user provides an instruction trace in the format given above. The framework then generates the corresponding pair of execution traces. This mode enables rapid construction and analysis of custom scenarios.
	\item \textbf{Random search}: The framework generates random instruction traces within user-defined constraints and checks the resulting execution traces against a specified property. For example, it can explore all traces of length 5 containing one branch instruction and at most two RAW dependencies. While this method cannot guarantee exhaustive coverage of the state space, it is effective for quickly finding counterexamples in large spaces.
	\item \textbf{State exploration}: The trace template is specified as a generator function, similar to random search mode. The framework then exhaustively verifies the property on every possible input, ensuring complete state space coverage. This mode is useful for proving properties about the model, but may be inefficient for finding counterexamples in large spaces due to the potential for excessive exploration of uninteresting subspaces.
\end{enumerate}


In summary, we created a tool capable of studying traces both in automated and guided way. Our time-efficient implementation enables exploration of significantly larger state spaces that are infeasible to analyze using Binder's original framework. While TLA$^+$ offers greater expressive power for formalizing properties such as leveraging temporal logic, in the context of Binder et al., the verified property was ultimately specified as a state predicate embedded in Python code. Therefore, we believe that our choice of implementation does not result in a substantial loss of expressiveness or rigor for the intended analyses.

\section{Generating TA Examples}

TODO: describe the setting, the search space given to the tool to produce an example (comment on performance "example was found in ... ms", "... examples were found"). Explain why examples show TA

We start with generating representative examples of branch-caused TA. The hypothesis is that the correct branch prediction may yield a longer execution than the same setting with misprediction. We want to have a simple example, thus decided we perform an exhaustive search in a state of the programs with:

\begin{enumerate}
    \item 4 committed instructions with;
    \item At most 2 dependencies;
    \item 1 branch instruction.
\end{enumerate}

The latency of branch instruction was chosen to be $1$ as conditionals jumps often require simple one-cycle operations (such as equality, more or equal, less or equal, etc.). The latencies of other instructions in the chosen setting are $4$. We have chosen single-scalar pipeline with 2 FUs.

Our framework explored $4608$ input traces in around $150$ ms: among those 2 TAs were identified, explained in Examples \ref{ex:bp-ta} and \ref{ex:bp-ta-1}.

\begin{example}
Figure \ref{fig:bp-ta-inputs-0} shows an instruction trace found from generating random examples. $C$ is a branch with misprediction region ${D, E}$. There is one dependency: $A \rightarrow B$. Figure \ref{fig:bp-ta-traces-0} shows the associated pair of execution traces. In trace $\alpha$ instructions $D$ and $E$ as skipped due to the correct prediction, so $F$ is fetched right away. This causes an earlier execution of $F$, which leads to $FU2$ being busy at clock cycle $7$ and therefore instruction $B$ starts being executed later, thus causing a slowdown.

\label{ex:bp-ta}
\end{example}

\begin{example}
The other TA which input and execution trace are shown in Figures \ref{fig:bp-ta-inputs-1} and \ref{fig:bp-ta-traces-1} respectively is different from Example \ref{ex:bp-ta} only by the FU of instruction $C$ while the other instructions are the same (also the misprediction region is longer due to longer delay between branch prediction and branch resolution).
\label{ex:bp-ta-1}
\end{example}


\begin{figure}[H]
    \centering
    \begin{subfigure}[b]{0.45\textwidth}
        \centering
        \begin{tabular}{rr|ccc}
            &  & Res. & Dep. & Lat. \\ \hline
            \textit{A:} &  & FU1 &  & $4$ \\
            \textit{B:} &  & FU2 & $\{A\}$ & $4$ \\
            \textit{*C:} &  & \textbf{FU2} &  & $1$ \\
            & \textit{D:} & FU1 &  & $4$ \\
            & \textit{E:} & FU1 &  & $4$ \\
            \textit{H:} &  & FU2 &  & $4$ \\
        \end{tabular}
        \caption{Input from Example \ref{ex:bp-ta}}
        \label{fig:bp-ta-inputs-0}
    \end{subfigure}
    \hfill
    \begin{subfigure}[b]{0.45\textwidth}
        \centering
        
        \begin{tabular}{rr|ccc}
            &  & Res. & Dep. & Lat. \\ \hline
            \textit{A:} &  & FU1 &  & $4$ \\
            \textit{B:} &  & FU2 & $\{A\}$ & $4$ \\
            \textit{*C:} &  & \textbf{FU1} &  & $1$ \\
            & \textit{D:} & FU1 &  & $4$ \\
            & \textit{E:} & FU1 &  & $4$ \\
            & \textit{F:} & FU1 &  & $4$ \\
            & \textit{G:} & FU1 &  & $4$ \\
            \textit{H:} &  & FU2 &  & $4$ \\
        \end{tabular}
        \caption{Input from Example \ref{ex:bp-ta-1}}
        \label{fig:bp-ta-inputs-1}
    \end{subfigure}
    \caption{Two anomalous inputs found from the setting}
    \label{fig:bp-ta-inputs}
\end{figure}



\begin{figure}[H]
    \centering
    \begin{subfigure}[b]{0.49\textwidth}
        \centering
        \includegraphics[width=\textwidth]{figures/simple-branch-ta.png}
        \caption{Trace from Example \ref{ex:bp-ta}}
        \label{fig:bp-ta-traces-0}
    \end{subfigure}
    \hfill
    \begin{subfigure}[b]{0.49\textwidth}
        \centering
        \includegraphics[width=\textwidth]{figures/simple-branch-ta-1.png}
        \caption{Trace from Example \ref{ex:bp-ta-1}}
        \label{fig:bp-ta-traces-1}
    \end{subfigure}
    \caption{Two TA traces found by the framework}
    \label{fig:bp-ta-traces}
\end{figure}

The only essential difference between Examples \ref{ex:bp-ta} and \ref{ex:bp-ta-1} is the length of misprediction region. The scheduling of instructions $A$, $B$ and $H$ is exactly the same in the two examples. This gives us a hint that some anomalies may fall down in the same category which can give us a classification of TAs.

Another notable observation is that here the anomalous effect can be explained by just by the later fetch of instruction $H$. Interestingly, the same effect can be obtained by modeling a cache miss for the fetch of $H$ as shown in Figure \ref{fig:equiv-to-bp-ta}. Here, int both $\alpha$ and $\beta$ the prediction is correct, however, there is a cache hit in $\alpha$ and cache miss in $\beta$.

TODO: TA only on commit; no longer effect?

\begin{figure}[H]
    \centering
    \includegraphics[width=\textwidth]{figures/equiv-trace.png}
    \caption{Cache miss on fetch of $H$ causing the same effect as branch misprediction}
    \label{fig:equiv-to-bp-ta}
\end{figure}

\section{Formalizing Definition}

TODO: start with latency, show it on example, formalize an assumption: show new contradicting examples

For ex 2 3 BInder's def works ...

but what dfo we do if smth inside of spec region?

\TODO{introdcue letters for instructions to make things clear}

Since there exist an equivalent behavior (shown at Figure \ref{fig:equiv-to-bp-ta}) to Examples \ref{ex:bp-ta}  and \ref{ex:bp-ta-1}, it should be possible to adapt Binder's definition for these cases. Let us take an Example \ref{ex:bp-ta}. In Binder's definition the source of TA is a variation, thus it must be defined first. In trace showing equivalent behavior (Figure \ref{fig:equiv-to-bp-ta}) the variation is in IF latency i.e. the delay between $\IFa$ and $\IFr$. In the Example \ref{ex:bp-ta} both $\IFa$ and $\IFr$ are different between traces $\alpha$ and $\beta$ thus the same variation can not be used. We may use the fact that for a given instruction $\IFa$ coincides with $\IFr$ of previous instruction and use this $\IFr$ as the start of the latency. Thus, the latency can be measured between $\IFr$ of branch instruction and $\IFr$ of first instruction after misprediction region.

However, we still need to take into account other variations, so the possible variation in IF of the after-branch instruction will be mixed with branch-related variation as their latencies overlap. So instead we propose using two events to signify the variation:

\begin{enumerate}
    \item \textbf{Branch Prediction (BP)} -- moment when prediction happens and speculative executions starts;
    \item \textbf{Correct Branch Taken (BT)} -- the instruction from correct branch enters the pipeline.
\end{enumerate}

Note that $BP$ corresponds to $\IFr$ of branch instruction and $BT$ is $\IFa$ of a first instruction after misprediction region of the branch. In case of a single variation that is in branch prediction, the TA pattern is detected in the same way as shown in Figure \ref{fig:equiv-to-bp-ta}, but with latency being shorter by 1 clock cylce (\TODO{or more...}) and the delay being longer by the same value.

Figure \ref{fig:bp-ta-analysed} shows this idea applied to Example \ref{ex:bp-ta}. Both latencies corresponding to variation are 1 clock cycle shorter compared to example shown in Figure \ref{fig:equiv-to-bp-ta}. The causality region also starts 1 clock cycle earlier, however is still going through the same events. Note here that we do not need any additional rules to build causality arcs.

A major assumption that allowed us to adapt the definition is that the branch behavior is just postponing the fetch of instruction following misprediction region. In this case the behavior can be compared to the effect of cache miss in IF stage -- this is also not completely true, as the penalty for cache miss is fixed while the size of misprediction region depends on the time between the prediction and the resolution of the branch. Therefore, in the rest of the article we focus more on examples that do not fit the assumption.

\TODO{Bound the effect of branch resolution?}

\TODO{Conclusion: definition somehow works, but we need to study its limitations}

\begin{figure}[H]
    \centering
    \includegraphics[width=\textwidth]{figures/simple-branch-ta-analysed.png}
    \caption{\TODO{}}
    \label{fig:bp-ta-analysed}
\end{figure}

\section{Limitations of the Definition}

So far we have demonstrated the potential of using Binder's definition in detecting branch prediction-related TAs. Now we discuss the limitations of the method by providing the relevant examples. We start from describing the problem that we found that appears within the existing framework and is not even related to branch prediction. Then we give more complex examples of branch prediction-caused TAs and try to reason whether they exhibit the same type of flaw or some new issues.

\subsection{Gap Problem}

While testing an existing definition using framework made by Binder et al. we encountered a situation when TA clearly exists but is not captured by the framework. Such an issue we called the \textit{gap problem}. Figure \ref{fig:gap-problem} shows an example of such a problem: pair of traces $\alpha/\beta$ shows and example of TA similar one seen in Example \ref{ex:simple-ta}, Binder's definition works fine for this pair. In a pair of traces $\alpha'/\beta'$ we change a latency for fetch of instruction $B$. This creates a so-called \textit{gap} between $\FUr$ of $A$ and $\FUa$ of $B$: a situation where a ETDG arc does not become a causality arc. Here it happens because a more strong causality link $(\IFr_B) \rightarrow (\FUa_B)$ is established using the rule of the most relevant constraint. Therefore, there is no causality path between end of the variation $\FUa_A$ and $COM_D$, so the timing anomaly is not signalled.

However, one can argue, that TA exists in $\alpha'/\beta'$ as we observe a slowdown in the trace $\alpha'$ with a favorable variation. Moreover, the two pairs of traces share the same suffixes and the scheduling pattern that leads to TA is the same. The only thing that prevents the TA from being detected is the absence of causality path between $\FUr$ of $A$ and $\FUa$ of $B$ in trace $\alpha'$.

The understanding of the flaws of a base definition allows us to reason about further results in terms of gap problem: following we describe some examples of branch prediction TA examples that do not fit into causality-based approach and identify whether it is the same type of problem or not.

\begin{figure}[H]
    \centering
    \includegraphics[width=\textwidth]{figures/gap-problem.png}
    \caption{Gap problem}
    \label{fig:gap-problem}
\end{figure}


\subsection{Early FU release}

So far we were considering branch predictor related TA caused by postponing the fetch of first non-mispredicted instruction.  This is possible because no instruction is in execution phase in misprediction region and therefore a region has no effect on scheduling on FU. In subsequent examples we analyze the effect of misprediction region on instructions fetched before the branch.

\begin{example}
\textbf{Nested Misprediction Region}

We consider an instruction trace from Figure \ref{fig:nested-bp-ta-trace} which consists of two nested misprediction regions: of instructions $C$ and $D$ respectively. We explore the variation in the prediction of $D$ -- in trace $\alpha$ (Figure \ref{fig:nested-bp-ta-trace}) it is correct yielding an overall longer execution time than in trace $\beta$ where the prediction is incorrect.
    
\label{ex:nested-bp-ta}
\end{example}


\begin{figure}[H]
    \centering
    \begin{tabular}{rrr|ccc}
    &  &  & Res. & Dep. & Lat. \\ \hline
    \textit{A:} &  &  & FU1 &  & $5$ \\
    \textit{B:} &  &  & FU2 & $\{A\}$ & $4$ \\
    \textit{C:} &  &  & FU1 &  & $1$ \\
    & \textit{*D:} &  & FU2 &  & $1$ \\
    &  & \textit{E:} & FU2 &  & $4$ \\
    &  & \textit{F:} & FU2 &  & $4$ \\
    & \textit{G:} &  & FU2 &  & $4$ \\
    \end{tabular}
    \caption{Instruction trace from example \ref{ex:nested-bp-ta}}
    \label{fig:nested-bp-ta-input}
\end{figure}


\begin{figure}[H]
    \centering
    \includegraphics[width=\textwidth]{figures/nested-bp-ta.png}
    \caption{Execution trace of Example \ref{ex:nested-bp-ta}}
    \label{fig:nested-bp-ta-trace}
\end{figure}


In Figure \ref{fig:nested-bp-ta-trace} we try to construct a causality graph for trace $\alpha$. On the picture we show the causal path form variation to the commit of $C$. However, one causal arc on the path (encircled in red) does not fit in the definition: Remember that in Binder's definition we have the \textit{rule 4} which creates a ETDG-arc $(\FUa \xrightarrow{lat_{FU}} \FUr)$. In Example \ref{ex:nested-bp-ta} the FU-latency of $G$ is $4$, however in trace $\alpha$ the delay between $\FUa_G$ and $\FUr_G$ is only 2 clock cycles because of the earlier release of FU2 as the result of squashing. So according to the Binder's definition the arc cannot be established and not TA exists.

Likely, we are just missing a rule that suffices to describe anomaly here. Intuitively, $\FUr_G$ is caused by the squashing by the branch resolution. Therefore, $(\FUa_C \xrightarrow{0} \FUr_G)$ arc would make sense (Assumption \ref{ass:FU-squashing}). However, it does not allow to construct a causality path in Example \ref{ex:nested-bp-ta} that can explain the TA happening. 

\begin{assumption}
If FU is released as the result of squashing, the FU release of corresponding branch is causal to FU release of the squashed instruction.
\label{ass:FU-squashing}
\end{assumption}

Indeed, what we see in this example is that the earlier fetch of $G$ allows it to start executing, which prevents $B$ from starting the execution earlier. In fact, it is a reordering problem similar to Example \ref{ex:bp-ta}. For that reason we take an Assumption \ref{ass:FU-acq-rel} that effectively fills the gap in the causality path.

\TODO{why this is not a gap problem?}

\begin{assumption}
The FU acquisition is always causal to the respective FU release.
\label{ass:FU-acq-rel}
\end{assumption}

However, the following example shows that Assumption \ref{ass:FU-acq-rel} is not always correct. 

\begin{example}
\textbf{Latency Impact Through Misprediction Region}
We consider an instruction trace from Figure \ref{fig:lat-mispred-ta-input} the pair of execution traces with TA is shown at Figure \ref{fig:lat-mispred-ta-trace}. Here the variation is in latency of $B$, but the effect propagates through misprediction region of $F$ which consists of one instruction $G$. 
\label{ex:lat-mispred-ta}
\end{example}


\begin{figure}[H]
    \centering
    \begin{tabular}{rr|ccc}
    &  & Res. & Dep. & Lat. \\ \hline
    \textit{A:} &  & FU3 &  & $9$ \\
    \textit{B:} &  & FU1 &  & $6|7$ \\
    \textit{C:} &  & FU2 & $\{A, B\}$ & $4$ \\
    \textit{D:} &  & FU1 & $\{C\}$ & $4$ \\
    \textit{E:} &  & FU2 & $\{B\}$ & $4$ \\
    \textit{F:} &  & FU1 &  & $1$ \\
    & \textit{G:} & FU2 &  & $4$ \\
    \end{tabular}
    \caption{Instruction trace of Example \ref{ex:lat-mispred-ta}}
    \label{fig:lat-mispred-ta-input}
\end{figure}


\begin{figure}[H]
    \centering
    \includegraphics[width=\textwidth]{figures/lat-mispred.png}
    \caption{Execution trace of Example \ref{ex:lat-mispred-ta}}
    \label{fig:lat-mispred-ta-trace}
\end{figure}

Example \ref{ex:lat-mispred-ta} illustrates the problem with Assumption \ref{ass:FU-acq-rel}:  here if we try to draw a causality arc $(\FUa_G \xrightarrow{2} \FUr_G)$ in trace $\alpha$, the causality path can not build between $FUr_B$ and $COM_E$. Instead, the Assumption \ref{ass:FU-squashing} works here allowing us to draw an arc $(\FUr_F \xrightarrow{0} \FUr_G)$.

\TODO{problem: consider only one trace}

\TODO{Conclusion: we need another definition}

\begin{example}
\textbf{Branch Prediction + Gap problem}

We modify Example \ref{ex:nested-bp-ta} by increasing latencies of instructions $A$ and $C$ by 2 (input trace in Figure \ref{fig:bp-gap-input}). This also requires us to define a larger misprediction region. The resulting pair of traces (Figure \ref{fig:bp-gap-trace}) exhibit a similar TA: instruction $J$ is fetched in trace $\alpha$ and postpones $B$.

\label{ex:bp-gap}
\end{example}

\begin{figure}[H]
    \centering
    \begin{tabular}{rrr|ccc}
    &  &  & Res. & Dep. & Lat. \\ \hline
    \textit{A:} &  &  & FU1 &  & $7$ \\
    \textit{B:} &  &  & FU2 & $\{A\}$ & $4$ \\
    \textit{C:} &  &  & FU1 &  & $1$ \\
    & \textit{*D:} &  & FU2 &  & $3$ \\
    &  & \textit{E:} & FU2 &  & $4$ \\
    &  & \textit{F:} & FU2 &  & $4$ \\
    &  & \textit{G:} & FU2 &  & $4$ \\
    &  & \textit{H:} & FU2 &  & $4$ \\
    & \textit{I:} &  & FU1 &  & $4$ \\
    & \textit{J:} &  & FU2 &  & $4$ \\
    \end{tabular}    

    \caption{Instruction trace from Example \ref{ex:bp-gap}}
    \label{fig:bp-gap-input}
\end{figure}

\begin{figure}[H]
    \centering
    \includegraphics[width=\textwidth]{figures/bp-gap.png}
    \caption{Execution trace from Example \ref{ex:bp-gap}}
    \label{fig:bp-gap-trace}
\end{figure}

Example \ref{ex:bp-gap} shows that the gap problem can also arise in misprediction region. $\FUr_J$ in trace $\alpha$ is not detected as a part of causal region of the end of the variation $\IFa_I$ because of the resource contention rule which creates a stronger arc $(\FUr_D \xrightarrow{0} \FUa_J)$.

In summary, our analysis demonstrates that while the definition of Binder et al. shows promise for application to branch prediction scenarios, its adaptation requires the introduction of additional assumptions. However, the two assumptions we considered are mutually contradictory, precluding the formulation of a consistent and comprehensive definition. Furthermore, the original definition is susceptible to the gap problem, which also manifests in the context of branch prediction-induced timing anomalies.

We contend that these issues stem from fundamental limitations in the underlying notion of causality employed by the framework. Both the gap problem and the contradictions between the proposed assumptions appear to arise from the same conceptual shortcomings. Consequently, in the following section, we propose a novel approach to causality that aims to address these issues.

\section{Towards New Causality Definition}

TODO: formalize properties that we expect from causality graph and explain why they are violated

TODO: propose a method that can be used to construct a "true causality" graph: set of constraints, moving events around
\chapter{Conclusion}
In this study, we developed a framework to investigate branch prediction-induced timing anomalies (TAs). This framework enables the construction of representative examples, allowing us to systematically explore the phenomenon. Utilizing this tool, we demonstrated that the definition proposed by Binder et al., which we identified as the most prominent among existing definitions, can be adapted to scenarios involving branch prediction. Nevertheless, our analysis revealed controversial cases that challenge whether Binder's notion of causality truly captures the intuitive understanding of causality in this context. Consequently, we conclude that further investigation into this aspect is necessary.

There are several features that we did not implement in the current framework, such as a shared memory bus model and limitations on the reservation station (RS) and reorder buffer (ROB). These extensions could be incorporated with minimal changes to the framework. Additionally, we did not address the effects of multiple TAs or how effectively their impacts can be separated using the current definition. However, given the lack of a consistent definition, it is premature to pursue this direction.

Another important aspect not addressed by our model is the state of the branch predictor. In our approach, prediction and misprediction are treated as black-box events, similar to how we abstract cache-induced latencies as non-deterministic. While this abstraction simplifies the model, it overlooks the potential to explicitly represent the branch predictor state. As discussed in Chapter~\ref{chap:proc-arch}, branch predictor requires additional hardware features, such as Pattern History Tables (PHT) and Branch Target Buffers (BTB). They could be modeled as part of the pipeline state, allowing us to consider related events in greater detail. This extension could provide deeper insights into how branch decisions propagate through the system and enable us to study not only individual prediction or misprediction events, but also sequences of such events, potentially revealing new types of timing anomalies.

%=========================================================


%=========================================================
\backmatter

\bibliographystyle{plain} % plain-fr si rapport en français 
\bibliography{bibfile.bib}

% \include{./chapters/7_Appendix}

%\cleardoublepage % Goes to an odd page
%\pagestyle{empty} % no page number
%~\newpage % goes to a new even page

\end{document}