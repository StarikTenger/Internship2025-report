\documentclass{beamer}
\usetheme{Copenhagen}
\usepackage{hyperref}
\usepackage[T1]{fontenc}
% \usepackage{enumitem}
\usepackage{xcolor}
\usepackage{pifont}

% other packages
\usepackage{latexsym,amsmath,xcolor,multicol,booktabs,calligra}
\usepackage{graphicx,pstricks,listings,stackengine}

% Define custom symbols
\newcommand{\cmark}{\textcolor{green!70!black}{\ding{51}}} % green tick
\newcommand{\xmark}{\textcolor{red}{\ding{55}}}            % red cross

\newenvironment{greenblock}[1]{%
  \setbeamercolor{block title alerted}{fg=white,bg=green!70!black}%
  \setbeamercolor{block body alerted}{fg=black,bg=green!20}%
  \begin{alertblock}{#1}
}{%
  \end{alertblock}
}

\author{\textbf{Andrei Ilin} \\ Supervised by Lionel Rieg, Florian Brandner and Mihail Asavoae}
% \author{Supervised by Lionel Rieg, Florian Brandner and Mihail Asavoae}
\title{Timing Anomaly through Branch Prediction}
% \subtitle{Cospa group meeting}
\subtitle{}
\institute{Université Grenoble Alpes}
\date{23 June 2025}
\usepackage{USTC_beamer}

% \renewcommand{\familydefault}{\rmdefault}

% defs
\def\cmd#1{\texttt{\color{red}\footnotesize $\backslash$#1}}
\def\env#1{\texttt{\color{blue}\footnotesize #1}}
\definecolor{deepblue}{rgb}{0,0,0.5}
\definecolor{deepred}{rgb}{0.6,0,0}
\definecolor{deepgreen}{rgb}{0,0.5,0}
\definecolor{halfgray}{gray}{0.55}

\lstset{
    basicstyle=\ttfamily\small,
    keywordstyle=\bfseries\color{deepblue},
    emphstyle=\ttfamily\color{deepred},  
    stringstyle=\color{deepgreen},
    numbers=left,
    numberstyle=\small\color{halfgray},
    rulesepcolor=\color{red!20!green!20!blue!20},
    frame=shadowbox,
}

% \includeonlyframes{current}
\begin{document}

% Define custom item styles


% \kaishu
\renewcommand{\figurename}{Fig.}

% logo
\begin{frame}[label=current]
    \titlepage
    \begin{figure}[htpb]
        \begin{center}
            \includegraphics[width=0.2\linewidth]{pic/logo-uga.png}\hspace{1.5cm}
            \includegraphics[width=0.2\linewidth]{pic/logo-verimag.png}\hspace{1.5cm}
            \includegraphics[width=0.2\linewidth]{pic/logo-INP.png}
        \end{center}
    \end{figure}
\end{frame}
% \TODO fix title

% \begin{frame}
%     \tableofcontents[sectionstyle=show,subsectionstyle=show/shaded/hide,subsubsectionstyle=show/shaded/hide]
% \end{frame}

\section{Introduction}

\begin{frame}{Critical Real-Time Systems}
    \begin{columns}
        \begin{column}{0.5\textwidth}
            \begin{block}{Can be found in:}
                \begin{itemize}
                    \item Cars 
                    \item Planes
                    \item Life-supporting equipment
                \end{itemize}
            \end{block}
        \end{column}

        \begin{column}{0.5\textwidth}
            \includegraphics[width=\textwidth]{pic/plane.png}
        \end{column}
    \end{columns}

    \hfill \break
    \hfill \break

    Strict timing requirements for the programs.

    \begin{exampleblock}{Example}
        Car break system that must respond in 5 ms.
    \end{exampleblock}
    
\end{frame}

% Ex: break respond time...

\begin{frame}{WCET Analysis}
    \begin{columns}
        
    \column{0.4\textwidth}

    \textbf{W}orst \textbf{C}ase \textbf{E}xecution \textbf{T}ime Analysis:
    \begin{itemize}
        \item Hardware + Software
        \item Upper bound for execution time?
    \end{itemize}


    \begin{block}{}
        Analysis is split into multiple stages and uses simplifications.
    \end{block}

    \column{0.6\textwidth}

    \includegraphics[width=1\textwidth]{pic/WCET.png}

    \includegraphics[width=1\textwidth]{pic/wcet-deps.png}

    \end{columns}
\end{frame}

% Animate distribution?
% Say: overestimation, simplification 
% Full system too complicated -> split anal. -> ASS.: sys are composable
% Not all are composable, thus TA


\begin{frame}{Timing Anomalies}
    \begin{block}{Timing Anomaly (TA)}
        When a local speedup leads to a global slowdown.
    \end{block}

    \begin{exampleblock}{Example}
        Faster completion of $A$ leads to a slowdown of the whole trace. 
    \end{exampleblock}

    \begin{columns}
    \column{0.4\textwidth}
        \includegraphics[width=1\textwidth]{pic/first-TA-ex-input.png}
    \column{0.6\textwidth}
        \includegraphics[width=1\textwidth]{pic/first-TA-ex-trace.png}
    \end{columns}
\end{frame}

% 

\section{Hardware}

\begin{frame}{OoO Multiscalar Pipeline}
    Processor:
    \begin{itemize}
        \item \textbf{Pipelined:} Divided into consecutive stages
        \item \textbf{Superscalar:} Fetch multiple instruction in the same time
        \item \textbf{Out-of-Order:} Execution order dictated by scheduling policy
    \end{itemize}


    \begin{columns}
    \column{0.5\textwidth}
        \includegraphics[width=\textwidth]{pic/pipeline.png}
    \column{0.5\textwidth}
        \includegraphics<1>[scale=0.14]{pic/multiscalar-demo-1.png}
        \includegraphics<2>[scale=0.14]{pic/multiscalar-demo-2.png}
        \includegraphics<3>[scale=0.14]{pic/multiscalar-demo-3.png}
        \includegraphics<4>[scale=0.14]{pic/multiscalar-demo-4.png}
    \end{columns}

\end{frame}

% Explain ROB: keep track of dependencies, each instr = 1 entery in ROB, when enters
% No TA: just one trace: some arrows clarif.


\begin{frame}{Branch Prediction}
    
    \begin{itemize}
        \item<1-3> Branch target unknown until branch is resolved.
        \item<2-3> Predict next address and execute speculatively.
        \item<3> Misprediction detected $\rightarrow$ start fetching the correct branch
    \end{itemize}

    
    \includegraphics<1>[scale=0.17]{pic/bp-demo-no.png}
    \includegraphics<2>[scale=0.17]{pic/bp-demo-correct.png}
    \includegraphics<3>[scale=0.17]{pic/bp-demo-incorrect.png}

\end{frame}

% \TODO Show first without BP: we have to wait; Animation




\section{Timing Anomalies}

\begin{frame}{Formal Definition?}
    \textbf{How do we formally define a TA?}

    \only<1>{
        \begin{block}{Many existing definitions}
        \begin{itemize}
                \item Step Heights (by Gebhard et al.)
                \item Step-functions Intersections (by Cassez et al.)
                \item Component Occupation (by Kirner et al.)
                \item Instruction Locality (by Reineke et al.)
                \item Event Time Dependency Graph (by Binder et al.)
            \end{itemize}
        \end{block}
    }

    \only<2>{
        \begin{block}{Many existing definitions}
        \begin{itemize}
                \item \textbf{Step Heights (by Gebhard et al.)}
                \item \textbf{Step-functions Intersections (by Cassez et al.)}
                \item Component Occupation (by Kirner et al.)
                \item Instruction Locality (by Reineke et al.)
                \item \textbf{Event Time Dependency Graph (by Binder et al.)}
            \end{itemize}
        \end{block}
    }
    
\end{frame}

% Key ideas of defs?

\begin{frame}{Step Heights by Gebhard et al.}
    \begin{block}{Definition}
        TA = instruction latency (compared to previous instr.) smaller, global time greater
    \end{block}

    \includegraphics<1>[width=\textwidth]{pic/step-height-good.png}

    \begin{alertblock}<2>{Counterexample!}
        \includegraphics[width=0.7\textwidth]{pic/step-height-bad.png}
    \end{alertblock}
\end{frame}
% Be simpler: arrow in diff. directions
% Counterexamples: 2 lines cross, but no TA!
% Swap coutnerexamples !!!


\begin{frame}{Step Function Intersections by Cassez et al.}
    \begin{block}{Definition}
        TA = one instruction finishes earlier than in alternative trace, the other later than in alternative trace
    \end{block}

    \only<1>{
        \begin{columns}
        \column{0.8\textwidth}
            \includegraphics[width=\textwidth]{pic/step-height-good.png}
        % \column{0.4\textwidth}
        %     \includegraphics[width=\textwidth]{pic/step-functions.png}
    \end{columns}
    }
    
    \only<2>{
        \begin{alertblock}{Counterexample!}
            \includegraphics[width=0.9\textwidth]{pic/step-functions-bad.png}
        \end{alertblock}
    }   
\end{frame}
% Idea of rotating the exec. pattern
% Counterexample: step function !!
% Is example correct?

\begin{frame}{Event Time Dependency Graph by Binder et al.}
    \begin{block}{3 Components}
        \only<1> {
            \begin{enumerate}
                \item Variation
                \item Slowdown
                \item Causality
            \end{enumerate}
        }
        \only<2> {
            \begin{enumerate}
                \item \textbf{Variation} in latency
                \item Slowdown
                \item Causality
            \end{enumerate}
        }
        \only<3> {
            \begin{enumerate}
                \item Variation in latency
                \item \textbf{Slowdown} in delay end of variation $\rightarrow$ later event
                \item Causality
            \end{enumerate}
        }
        \only<4> {
            \begin{enumerate}
                \item Variation in latency
                \item Slowdown  in delay end of (variation $\rightarrow$ later event)
                \item \textbf{Causality} link = event $X$ cannot be earlier because of $Y$
            \end{enumerate}
        }
    \end{block}

    \includegraphics<1>[scale=0.17]{pic/binder-def-1.png}
    \includegraphics<2>[scale=0.17]{pic/binder-def-2.png}
    \includegraphics<3>[scale=0.17]{pic/binder-def-3.png}
    \includegraphics<4>[scale=0.17]{pic/binder-def-4.png}
\end{frame}

% Causality - why do we need it? In counterex no relation between source and TA
% Causality only in one trace

\begin{frame}
    \begin{block}{Goal}
        Develop a consistent TA definition applicable for branch prediction.
    \end{block}

    \begin{itemize}
        \item \textbf{Question 1} Can timing anomaly be caused by branch prediction?
        \item \textbf{Question 2} Can we extend the existing Binder's definition?
        \item \textbf{Question 3} Do we cover all the aspects of branch prediction?
    \end{itemize}

\end{frame}
% + subquestions
% can we extend causality-based def?
% how to do that?

\section{Contribution}

\begin{frame}{Research Plan}
    \begin{enumerate}
        \item Create a tool for automatic example generation.
        \item Generate some candidate scenarios.
        \item Try to adapt definition on them.
        \item Find new scenarios to verify the definition.
    \end{enumerate}
\end{frame}
% Too long. Make it more visual
% TLA problems, C++ problems, compare

\begin{frame}{Tool Implementation}
    % \begin{alertblock}{TLA+ implementation}
    %     Is slow. 2-3 seconds to generate a single example.
    % \end{alertblock}

    % \begin{itemize}
    %     \item Written in C++.
    %     \item Simple and flexible input format.
    %     \item Supports speculative execution.
    %     \item Faster than TLA$+$ implementation (ms instead of seconds).
    %     \item 3 operation modes:
    %     \begin{enumerate}
    %         \item Interactive manual mode.
    %         \item Randomized search.
    %         \item State space exploration.
    %     \end{enumerate}
    %     \item Lacks formal gurantees
    % \end{itemize}

    

    \begin{columns}
        \column{0.5\textwidth}
        \begin{exampleblock}{Existing TLA$^+$ framework}

            \begin{itemize}
                \item[\xmark] Slow: 2-3 seconds for single example
                \item[\xmark] No branch and speculation support
                \item[\xmark] Lengthy input format 
                \item[\cmark] Formal guarantees
            \end{itemize}

            
        \end{exampleblock}

        \column{0.5\textwidth}
        \pause
        \begin{exampleblock}{Our C++ framework}
            \begin{itemize}
                \item[\cmark] Fast: a few ms per example
                \item[ ]
                \item[\cmark] Traces with speculation
                \item[ ] 
                \item[\cmark] Refined consize input format
                \item[\xmark] Harder to specify properties
            \end{itemize}
        \end{exampleblock}
    \end{columns}

\end{frame}
% Better to do a comparison: 2 columns, pros and cons
% We starte by doing TLA+ implem., but then...
% Animation

\begin{frame}{Input Format}

\begin{itemize}
    \item Each branch instruction is followed by \textbf{misprediction region} -- sequence of instruction representing the wrong branch.
    \item A pair of traces is generated from a single input.
\end{itemize}
% How the input format is new
% Meaning of indentation
% Explain squash on diagram, what diagram mean
% Write: some instr. disappera, so harder to compare

\begin{columns}
    \column{0.4\textwidth}
        \begin{tabular}{rr|ccc}
         &  & Res & Dep. & Lat. \\ \hline
        \textit{A} &  & FU1 &  & $4$ \\
        \textit{*B} &  & FU2 & $\{A\}$ & $1$ \\
        & \textit{C} & FU2 &  & $4$ \\
        & \textit{D} & FU2 &  & $4$ \\
        \textit{E} &  & FU2 &  & $4$ \\
        \end{tabular}

    \column{0.6\textwidth}
        \includegraphics[width=\textwidth]{pic/mispred-intro.png}
\end{columns}

\end{frame}
% \TODO add mispred notation

\begin{frame}{Branch TA}
    Correct prediction leads to a longer execution time.

    \includegraphics<1>[scale=0.13]{pic/simple-branch-ta-analyzed-1.png}
    \includegraphics<2>[scale=0.13]{pic/simple-branch-ta-analyzed-2.png}
    \includegraphics<3>[scale=0.13]{pic/simple-branch-ta-analyzed-3.png}
    \includegraphics<4>[scale=0.13]{pic/simple-branch-ta-analyzed-4.png}

    \begin{exampleblock}{Where is latency?}
        \only<2->{Between "branch prediction" and "correct branch taken" events.}
    \end{exampleblock}

    \only<3->{No need to adapt delay and causality.}

    
\end{frame}
% Question: can we fit Binder's def?
% ...and this is the answer
% Why latency
% TODO: animation

% Viusal example
% Names for assumtions!
% Restructure: introduce assumption with examples
% Put the input format!

% \begin{frame}{Early FU Release}
%     \begin{itemize}
%         \item \textbf{Assumption 1} does not hold.
%         \item \textbf{Assumption 2} holds.
%     \end{itemize}
%     \includegraphics[width=0.9\textwidth]{pic/nested-bp-ta.png}
% \end{frame}


\begin{frame}{FU release by FU acquisition}
    Example: when a new rule is required.

    \includegraphics<1>[scale=0.13]{pic/nested-bp-ta-1.png}
    \includegraphics<2>[scale=0.13]{pic/nested-bp-ta-2.png}
    \includegraphics<3>[scale=0.13]{pic/nested-bp-ta-3.png}
    \includegraphics<4>[scale=0.13]{pic/nested-bp-ta-4.png}
    \includegraphics<5>[scale=0.13]{pic/nested-bp-ta-5.png}
    \includegraphics<6>[scale=0.13]{pic/nested-bp-ta-6.png}

    \only<5> {
        \begin{alertblock}{ Binder's rule does not work}
           "4 cycles must be spent inside FU to establish causality."
        \end{alertblock}   
    }

    \only<6> {
        \begin{greenblock}{"Acquisition" Rule}
            The FU acquisition is always causal to the respective FU release.
        \end{greenblock}
    }
    
\end{frame}
% make it bigger

\begin{frame}{FU release by squashing}
    The problem with the "Acquisition" rule.
    \includegraphics<1>[scale=0.11]{pic/lat-mispred-1.png}
    \includegraphics<2>[scale=0.11]{pic/lat-mispred-2.png}
    \includegraphics<3>[scale=0.11]{pic/lat-mispred-3.png}
    \includegraphics<4>[scale=0.11]{pic/lat-mispred-4.png}

    \only<3> {
        \begin{alertblock}{"Acquisition" rule does not work}
           "The FU acquisition is always causal to the respective FU release."
        \end{alertblock}   
    }

    \only<4> {
        \begin{greenblock}{"Squashing" Rule}
            If FU is released as the result of squashing, the FU release is causal to branch resolve.
        \end{greenblock}
    }
\end{frame}
% trace shorter

\begin{frame}{Results}
    \begin{itemize}
        \item The definition of Binder et al. can be applied in branch prediction context with minimal adjustments.
        \item It is not clear how to modify the rule set to stay consistent.
    \end{itemize}
\end{frame}

% Say that we have a bigger problem

\begin{frame}{Gap Problem}
    Even the original definition by Binder et al. has a problem.
    \includegraphics<1>[scale=0.17]{pic/gap-problem-1.png}
    \includegraphics<2>[scale=0.17]{pic/gap-problem-2.png}
    \includegraphics<3>[scale=0.17]{pic/gap-problem-3.png}
    \begin{alertblock}<3>{To conclude:}
        The problem is in the causality notion itself.
    \end{alertblock}
\end{frame}
% Make an animation

% \begin{frame}{Results}
%     \begin{itemize}
%         \item The definition of Binder et al. can be applied in branch prediction context with minimal adjustments.
%         \item Sometimes it is not evident which rule to use.
%         \item Initial definition suffers from a gap problem.
%         \item Likely, the problem is with causality itself.
%     \end{itemize}
% \end{frame}

% Results about assumptions -> to assumptions


\section{Conclusion}

\begin{frame}{Conclusion}
    \textbf{What was Done:}
    \begin{itemize}
        \item Analyzed existing timing anomaly (TA) definitions; adopted Binder's causality-based approach.
        \item Developed a tool to systematically generate and analyze branch prediction-induced TAs.
    \end{itemize}

    \textbf{Results:}
    \begin{itemize}
        \item Binder's definition is adaptable, but controversial cases and a "gap problem" remain.
    \end{itemize}

    \textbf{Future Work:}
    \begin{itemize}
        \item A new causality definition based on event constraints to address these issues (work in progress).
        \item Study the impact of branch predictor state.
    \end{itemize}


\end{frame}
% Add structure: results, future work

\appendix

\section{Backup Slides}


\begin{frame}{Branch Prediction: possible implementations}
    \begin{columns}
    \column{0.5\textwidth}

    \begin{block}{Additional Hardware}
        \begin{itemize}
            \item Pattern History Table (PHT)
            \item Branch Target Buffer (BTB)
        \end{itemize}
    \end{block}

    \begin{exampleblock}{Example: 2-bit counter}
        For each branch store a 4-state automaton in PHT
    \end{exampleblock}

    \column{0.5\textwidth}

    \begin{block}{Strategies}
        \textbf{Static:} always take, never take, take backwards. \\        
        \textbf{Dynamic:} 
        \begin{itemize}
            \item 1 or 2-Bit Counter
            \item Global or Local History
            \item Global share
        \end{itemize}
        and more...
    \end{block}

    \end{columns}

    \includegraphics[width=0.9\textwidth]{pic/two-bit-counter.png}
\end{frame}


\end{document}
