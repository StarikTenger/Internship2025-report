\chapter{Conclusion}

In this report, we have analyzed several existing definitions of timing anomalies (TAs) and, after careful consideration, have chosen to proceed with the definition proposed by Binder et al.~\cite{binder_definitions_2022} as the basis for our study. Consequently, we developed a tool to investigate timing anomalies (TAs) induced by branch prediction. This tool enables the systematic construction and analysis of representative examples. Using this tool, we demonstrated that the definition proposed by Binder et al.~\cite{binder_definitions_2022}, based on variation and causality, and identified as the most prominent among existing definitions, can be adapted to scenarios involving branch prediction. However, our analysis revealed controversial cases that challenge whether Binder's notion of causality fully captures the intuitive understanding of causality in this context. Notably, one issue with Binder's definition, which we refer to as the gap problem, is present even in the original framework and is not specific to branch prediction only. Therefore, further investigation into the aspect of causality is required.

We are currently developing an alternative causality definition based on the notion of constraints for events in the trace. Our approach interprets a change in a constraint as a variation and derives causality links from the resolution procedure, which incrementally adapts the trace to updated constraints. Preliminary evidence suggests that this approach addresses the gap problem.

Also, there are several features that we did not implement in the current tool, such as a shared memory bus model and limitations on the reservation station (RS) and reorder buffer (ROB). These extensions could be incorporated with minimal changes to the tool. Additionally, we did not address the effects of multiple TAs or how effectively their impacts can be separated using the current definition. However, given the lack of a consistent definition, it is premature to pursue this direction.

Another important aspect not addressed by our model is the state of the branch predictor. In our approach, prediction and misprediction are treated as black-box events, similar to how we abstract cache-induced latencies as non-deterministic. While this abstraction simplifies the model, it overlooks the potential to explicitly represent the branch predictor state. As discussed in Chapter~\ref{chap:proc-arch}, branch predictor requires additional hardware features, such as Pattern History Tables (PHT) and Branch Target Buffers (BTB). They could be modeled as part of the pipeline state, allowing us to consider related events in greater detail. This extension could provide deeper insights into how branch decisions propagate through the system and enable us to study not only individual prediction or misprediction events, but also sequences of such events, potentially revealing new types of timing anomalies.
