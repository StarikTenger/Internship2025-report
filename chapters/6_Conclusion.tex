\chapter{Conclusion}
In this study, we developed a framework to investigate branch prediction-induced timing anomalies (TAs). This framework enables the construction of representative examples, allowing us to systematically explore the phenomenon. Utilizing this tool, we demonstrated that the definition proposed by Binder et al., which we identified as the most prominent among existing definitions, can be adapted to scenarios involving branch prediction. Nevertheless, our analysis revealed controversial cases that challenge whether Binder's notion of causality truly captures the intuitive understanding of causality in this context. Consequently, we conclude that further investigation into this aspect is necessary.

There are several features that we did not implement in the current framework, such as a shared memory bus model and limitations on the reservation station (RS) and reorder buffer (ROB). These extensions could be incorporated with minimal changes to the framework. Additionally, we did not address the effects of multiple TAs or how effectively their impacts can be separated using the current definition. However, given the lack of a consistent definition, it is premature to pursue this direction.

Another important aspect not addressed by our model is the state of the branch predictor. In our approach, prediction and misprediction are treated as black-box events, similar to how we abstract cache-induced latencies as non-deterministic. While this abstraction simplifies the model, it overlooks the potential to explicitly represent the branch predictor state. As discussed in Chapter~\ref{chap:foo}, different implementations of branch predictors, such as Pattern History Tables (PHT) and Branch Target Buffers (BTB), could be modeled as part of the pipeline state, allowing us to consider related events in greater detail. This extension could provide deeper insights into how branch decisions propagate through the system and enable us to study not only individual prediction or misprediction events, but also sequences of such events, potentially revealing new types of timing anomalies.
