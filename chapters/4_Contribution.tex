\chapter{Contribution}

Definition of Binder et al. \cite{binder_definitions_2022} is promising, however the branch prediction and the related issues are not taken in account by the framework. Thus we aim to extend the use case of proposed definition.

In our work we try to adjust Binder's definition to the setting of pipeline with branch predictor. We introduce an input format capable of expressing speculative execution. 

\TODO{complete intro when chapter is done}

\section{Methodology}

The implementation provided by Binder is written in TLA$^+$ \cite{lamport_specifying_2003}. The pipeline state is specified in set-theory notation. The model checker step corresponds to a one clock cycle and derives a new HW state from the previous one. This allows to simulate the non-deterministic timing behavior: each time when a variation can happen, multiple next state are generated. TLA$^+$ covers all reachable states ensuring that all possible behaviors are covered.

The pair of trace constitutes a whole model state. TA is expressed as an invariant for the pair of traces, so its is verified in each model checking step.

\TODO{each pair of executions is considered? or all executions are compared agains the one reference?}

\subsection{Input trace format}

The input of the framework is a pair of:
\begin{enumerate}
	\item Pipeline parameters: superscalar degree, $FU$ latencies and memory access latencies depending on the cache events (hit or miss). sequence of instructions;
	\item Instruction sequence: for each instruction its type and registers are specified as well as set of cache behaviors to be explored by the model checker. The type is used to know which $FU$ will be used by the instruction and based on registers data dependencies are retrieved.
\end{enumerate}

We can simplify this view by directly expressing the resource, dependencies and possible latencies of instruction.

give here an example



graph by python -> slow...

3 modes of using: manual, random, total


\section{Adapting definition of Binder et al.}



\section{Gap problem}

\section{Formal Requirements for Causality Graph}

\section{New Causality Definition}

\section{Taking BP state into account}
Put to conclusion?

\section{Results}

put examples of different TA here